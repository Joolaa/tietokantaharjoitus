\documentclass[a4paper, 12pt finnish]{article}
\usepackage[finnish]{babel}
\usepackage{graphicx}
\usepackage[utf8]{inputenc}
\usepackage[T1]{fontenc}

\title{Dokumentaatio \\ \large Tietokantasovellus}
\author{Joosua Laakso}
\date{\today}

\begin{document}
\maketitle

\section{Johdanto} 
\subsection{Työn aihe} Työn aiheena on työaikatietokanta. Sovellus on
suunniteltu erityisesti logistiikka-alan käyttöön. Tarkoituksena on, että
logistiikka-alalla työskentelevät autonkuljettajat voisivat paperisten
lomakkeiden täyttämisen sijaan syöttää tiedot työkeikkoihin käyttämästään
ajasta suoraan sovellukseen, jonka pitäisi huomattavasti vähentää 
työnantajan kirjanpitoasioihin käyttämää aikaa, koska tietoja ei enää
tarvitsisi kopioida käsin digitaaliseen muotoon paperisesta muodosta, vaan
ne olisi helposti saatavilla tietokannasta. Järjestelmään voi sisältyä
myös laskun automaattinen luonti, jonka avulla järjestelmään syötettyjen
tietojen perusteella, eli työntekijöiden syöttämien työaikatietojen ja 
työnantajan syöttämien tuntitaksatietojen, voisi luoda laskulomakkeen.

\subsection{Käytetyt tekniikat} Työ toteutetaan Helsingin yliopiston 
Users-palvelinta apuna käyttäen. Työssä käytetään Apache-palvelinsovellusta
sekä PHP-kieltä. Tietokannan hallintajärjestelmänä käytetään
Users-palvelimella paremmin tuettua PostgreSQL-järjestelmää.

\end{document}
